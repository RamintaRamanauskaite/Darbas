\documentclass[]{article}
\usepackage{lmodern}
\usepackage{amssymb,amsmath}
\usepackage{ifxetex,ifluatex}
\usepackage{fixltx2e} % provides \textsubscript
\ifnum 0\ifxetex 1\fi\ifluatex 1\fi=0 % if pdftex
  \usepackage[T1]{fontenc}
  \usepackage[utf8]{inputenc}
\else % if luatex or xelatex
  \ifxetex
    \usepackage{mathspec}
  \else
    \usepackage{fontspec}
  \fi
  \defaultfontfeatures{Ligatures=TeX,Scale=MatchLowercase}
\fi
% use upquote if available, for straight quotes in verbatim environments
\IfFileExists{upquote.sty}{\usepackage{upquote}}{}
% use microtype if available
\IfFileExists{microtype.sty}{%
\usepackage{microtype}
\UseMicrotypeSet[protrusion]{basicmath} % disable protrusion for tt fonts
}{}
\usepackage[left=3cm,right=3cm,top=2cm,bottom=2cm]{geometry}
\usepackage{hyperref}
\hypersetup{unicode=true,
            pdftitle={Lietuvos švietimo problemos},
            pdfauthor={Raminta Ramanauskaite},
            pdfborder={0 0 0},
            breaklinks=true}
\urlstyle{same}  % don't use monospace font for urls
\usepackage{longtable,booktabs}
\usepackage{graphicx,grffile}
\makeatletter
\def\maxwidth{\ifdim\Gin@nat@width>\linewidth\linewidth\else\Gin@nat@width\fi}
\def\maxheight{\ifdim\Gin@nat@height>\textheight\textheight\else\Gin@nat@height\fi}
\makeatother
% Scale images if necessary, so that they will not overflow the page
% margins by default, and it is still possible to overwrite the defaults
% using explicit options in \includegraphics[width, height, ...]{}
\setkeys{Gin}{width=\maxwidth,height=\maxheight,keepaspectratio}
\IfFileExists{parskip.sty}{%
\usepackage{parskip}
}{% else
\setlength{\parindent}{0pt}
\setlength{\parskip}{6pt plus 2pt minus 1pt}
}
\setlength{\emergencystretch}{3em}  % prevent overfull lines
\providecommand{\tightlist}{%
  \setlength{\itemsep}{0pt}\setlength{\parskip}{0pt}}
\setcounter{secnumdepth}{0}
% Redefines (sub)paragraphs to behave more like sections
\ifx\paragraph\undefined\else
\let\oldparagraph\paragraph
\renewcommand{\paragraph}[1]{\oldparagraph{#1}\mbox{}}
\fi
\ifx\subparagraph\undefined\else
\let\oldsubparagraph\subparagraph
\renewcommand{\subparagraph}[1]{\oldsubparagraph{#1}\mbox{}}
\fi

%%% Use protect on footnotes to avoid problems with footnotes in titles
\let\rmarkdownfootnote\footnote%
\def\footnote{\protect\rmarkdownfootnote}

%%% Change title format to be more compact
\usepackage{titling}

% Create subtitle command for use in maketitle
\providecommand{\subtitle}[1]{
  \posttitle{
    \begin{center}\large#1\end{center}
    }
}

\setlength{\droptitle}{-2em}

  \title{Lietuvos švietimo problemos}
    \pretitle{\vspace{\droptitle}\centering\huge}
  \posttitle{\par}
    \author{Raminta Ramanauskaite}
    \preauthor{\centering\large\emph}
  \postauthor{\par}
      \predate{\centering\large\emph}
  \postdate{\par}
    \date{2019 m birželis 19 d}

\usepackage[utf-8]{inputenc}
\\usepackage[L7x]{fontenc}
\usepackage[lithuanian,english]{babel}

\usepackage{setspace}
\onehalfspacing

\usepackage{hyperref}
\hypersetup{colorlinks=true, linkcolor=blue, filecolor=blue,
urlcolor=blue, citecolor=red}

\begin{document}
\maketitle

\begin{verbatim}
## Warning: package 'tidyverse' was built under R version 3.5.3
## Warning: package 'ggplot2' was built under R version 3.5.3
## Warning: package 'tibble' was built under R version 3.5.3
## Warning: package 'tidyr' was built under R version 3.5.3
## Warning: package 'readr' was built under R version 3.5.3
## Warning: package 'purrr' was built under R version 3.5.3
## Warning: package 'dplyr' was built under R version 3.5.3
## Warning: package 'stringr' was built under R version 3.5.3
## Warning: package 'forcats' was built under R version 3.5.3
## Warning: package 'rsdmx' was built under R version 3.5.3
## Warning: package 'reshape2' was built under R version 3.5.3
## Warning: package 'knitr' was built under R version 3.5.3
## Warning: package 'data.table' was built under R version 3.5.3
## Warning: package 'zoo' was built under R version 3.5.3
\end{verbatim}

Švietimas yra viena iš aktualiausių visuomenės gyvenimo sričių. Mūsų
šalyje ši sritis susiduria su problemomis. Daugelis pastebi, kad vyrauja
dideli mokinių pasiekimų skirtumai miestuose ir regionuose, mokytojai
išreiškia nusivylimą mažais atlyginimais, taip pat ir prieš metus įvestu
etatiniu apmokėjimu, valdžios institucijos atsigręžia į mokytojus
klausdami dėl silpnų mokinių rezultatų. Už švietimo kokybę atsakingi
asmenys gręžiojasi vieni į kitus ir ieško kaltų, kol situacija Lietuvoje
nesikeičia ir švietimo kokybė, mokytojų bei mokinių padėtis negerėja.
Taigi pradėkime nuo mokytojų problemų. Bene didžiausia problema, kurią
jie kelia -- atlyginimų dydis. 2016 metais išrinkus naują Lietuvos
Respublikos Seimą buvo paruošta ir LR Vyriausybės programa (Seimas
(2016)), kurioje vienas iš uždavinių -- įvesti pedagoginiam personalui
etatinį apmokėjimą. Visų pirma, svarbu išsiaiškinti, kuo ankstesnė
mokytojų darbo apmokėjimo tvarka buvo bloga (Profesąjungos (2017)).
Didžiausia problema -- labai dideli įkainiai kontaktinėms valandoms bei
mažas skaičius valandų pamokų pasiruošimui ir darbų vertinimui. Dėl
tokios tvarkos labiausiai nukentėdavo pedagogai, dirbantys mažų
miestelių ar kaimų mokyklose, kur neturėdavo daug kontaktinių valandų.
Spręsti šioms problemoms buvo pasitelktas etatinis apmokėjimas. Anot
buvusios LR Švietimo ir mokslo ministrės Jurgitos Petrauskienės,
„pagrindinis etatinės mokytojų darbo apmokėjimo sistemos tikslas --
užtikrinti aiškesnę etato struktūrą, suvokiant, kad mokytojų darbas yra
ne vien tik pamokų vedimas, bet kartu ir pasirengimas joms bei kitos
mokyklos bendruomenei naudingos veiklos`` (Židžiūnienė (2018)). Taip pat
ji teigia, kad ši nauja tvarka gali padėti sukurti patrauklesnę darbo
aplinką jauniems pedagogams ir užtikrinti teisingesnę darbo užmokesčio
sistemą. Tad kuo naujas mokytojų darbo apmokėjimas skiriasi nuo
senesnio? Anksčiau pedagogų atlyginimas labiausia priklausydavo nuo
turimų kontaktinių valandų skaičiaus. Kontaktinės valandos -- laikas,
per kurį mokytojas tiesiogiai dirba su mokiniais (pamokos, papildomojo
ugdymo pamokos, pamokos neformaliojo švietimo įstaigoje)( ŠMM (2011)).
Vadinasi, kuo daugiau buvo turima pamokų, tuo didesnį atlyginimą buvo
galima gauti. Ši tvarka ypač kenkdavo mažų miestelių ar kaimų
mokytojams, kurie mokyklose tiesiog negalėdavo turėti didelio
kontaktinio valandų skaičiaus ten, kur yra nedaug klasių, be to, tos
klasės dažnu atveju taip pat nėra didelės, nuo to taip pat kentėdavo
atlyginimo dydis. Tam tikrą atlyginimo dalį sudarydavo ir papildomos
valandos -- laikas, skirtas netiesioginiam darbui su mokiniais
(pasirengimas pamokoms, sąsiuvinių, kontrolinių darbų taisymas,
vadovavimas klasei ir kt.). Laikas mokinių darbų tikrinimui buvo
įvertintas maždaug 4 kartais mažesniu tarifu (priklauso nuo darbo stažo
bei kvalifikacinio laipsnio) ir tokių valandų buvo skiriama vos keletas
per savaitę. Vos kelios valandos buvo skirtos ir pasiruošimui pamokoms.
Už visus kitus papildomus darbus (bendravimui bei bendradarbiavimui su
tėvais, darbų planavimui, rūpinimuisi mokyklos renginiais) buvo
skiriamos 0,5 -- 3,5 valandos per savaitę. Šis skaičius priklausydavo
nuo mokyklos finansinių galimybių. Kas keitėsi po etatinio darbo modelio
priėmimo? Visų pirma, atlyginimai nebepriklauso nuo kontaktinių valandų.
Iš viso mokytojas per savaitę turi turėti 36 valandas (1512 valandų per
metus) veiklos (tiek kontaktinės, tiek ne kontaktinės valandos).
Mokytojai, dirbantys už 1 etatą, su didesniu nei 2 metai stažu per
savaitę gali turėti daugiausia 21 kontaktinę valandą (ne daugiau kaip
60\% viso etato). Pasiruošimui pamokoms bei vertinimui skiriamas valandų
skaičius lygus 40-60\% nuo kontaktinių valandų skaičiaus (priklauso nuo
dėstomo dalyko). 2,5 valandos per savaitę (102 valandos per metus)
vidutiniškai yra privalomai skirtos darbui įsitraukiant į mokyklos
bendruomenės veiklą, darbą su tėvais. Iš viso tokių valandų gali būti
502 per metus, arba kitaip sakant -- beveik trečdalis viso darbo
valandų. Etatinis darbo apmokėjimas leido sukurti šiek tiek geresnes
sąlygas pedagogams, neturintiems stažo. Jų kontaktinių valandų skaičius
per savaitę negali viršyti 18 valandų (per metus 756) ir tai yra iki
50\% etato. 55--80\% nuo kontaktinių valandų yra skirta pasiruošimui
pamokoms bei vertinimui (priklausomai nuo dėstomo dalyko).
Pradedantiesiems pedagogams taip pat privalomos 102 valandos darbo,
nesusijusio su pamokomis ar pasiruošimu joms, bet skiriamos mokyklos
bendruomenės veiklai. Tokiam etatiniam apmokėjimui netrūko kritikos.
Vienas iš aršiausių šios naujovės kritikų -- aukščiausią reitingą
Lietuvoje turinčios Vilniaus licėjaus gimnazijos direktorius Saulius
Jurkevičius (Jurkevičius (2018)). Jis akcentuoja, jog yra neteisinga
įkainoti vienodu tarifu pačias pamokas ir kitą veiklą, kitos veiklos
aprašymai užims daug laiko, norint kelti atlyginimus juos reikėtų kelti
atsižvelgiant į kontaktines valandas, reformos negalima įvesti birželio
viduryje, nes trūks laiko bei kad lėšos naudojamos neišmintingai, dėl
reformos nekils nei mokytojų algos nei bus pritraukta jaunų mokytojų.
Tad apie viską nuo pradžių. (ŠMM (2018))Turbūt tarifų suvienodinimu
labiausiai džiaugiasi tie mokytojai, kurie turėjo nedidelį kontaktinių
valandų skaičių. Taip, iš dalies tai naudinga ir reikalinga, bet ne
visais atvejais. Pamokos išaiškinimas mokiniams ko gero yra tik
mažesnioji pusė mokytojo įdėto darbo. Pamokoje matomas rezultatas būna
nulemtas, kiek laiko mokytojas ruošėsi pamokai dieną prieš jai
įvykstant. Sudėtingiausia yra jaunų mokytojų situacija, todėl nieko
nuostabaus, kad jiems suteikiama daugiau nekontaktinių valandų
pasiruošti pamokai nei stažą turintiems mokytojams. Tarkime, ką tik
studijas baigęs lietuvių kalbos mokytojas dirbantis gimnazijoje su 25
mokiniais kiekvienoje klasėje turi 17 kontaktinių valandų per savaitę,
t.y. 3 klases. Per dieną būtų apie 3--4 pamokas. Jam priklausytų 13,6
valandos pasiruošimui pamokoms bei jų vertinimui per savaitę, o
neskaičiuojant savaitgalių tai sudarytų 2,72 valandos per dieną (beveik
2 val. ir 45 min). Vienai pamokai pasiruošti per dieną tenka 41-55
minutės. Į šį laiką turėtų būti įskaičiuotas laikas pasiruošti teorinę
medžiagą (gerai, jei pamoka skirta gramatikai, tam reikia mažiau laiko,
o jeigu literatūrai -- papasakoti apie autorių, jo kūrinį,
istorinį/filosofinį/psichologinį/šių dienų kontekstą, ieškoti
informacijos, naujų straipsnių, juos paruošti pamokai bei
sukonspektuoti), užduotis, kurios padėtų mokiniams įtvirtinti žinias,
įvertinti praeitos pamokos rašto darbus/testus/rašinius, rezultatus
suvesti į dienyną. Jau nekalbant apie tai, kad geriausia neturėti
dvyliktokų, nes jų rašiniai kai kada siekia 700 žodžių, visus ištaisyti
ir parašyti kiekvienam komentarus užtruktų kelias valandas vienai klasei
per dieną. O jei dar mokinių ruošimas meninio skaitymo konkursams,
olimpiadoms\ldots{}. Tada suprantama, kad toks laiko tarpas skirtas
mokytojui yra tiesiog juokingai mažas, kaip ir apmokėjimas už tokį
darbą. Tačiau yra ir kitas mokytojų tipas, na, tarkim, kūno kultūros.
Jiems, kaip ir kitiems mokytojams priklauso valandos, skirtos pamokos
pasiruošimui bei įvertinimui. Pradedantis mokytojas turi 17 kontaktinių
valandų per savaitę. Jam priklausytų beveik 11 valandų per savaitę
pamokų pasiruošimui bei vertinimui. Bet juk kūno kultūroj kontrolinių
nėra. Ir testų nėra. Ir rašto darbų nėra (na, nebent labai jau reikia
pasitaisyti pažymį pusmetyje, na tada kokie 3 per visą pusmetį ryžtasi
tokiam žygdarbiui). Ir teorinės medžiagos ruošti nereikia. Na, programą
reikia perskaityti. Tai čia kiek, 10 min.? Na gerai, pusvalandis. Na
labai lėtai skaitant gal net valanda. Bet tai kur kitos 10 valandų per
savaitę??? Taigi galima daryti išvadą, kad valandos, paskirstytos
skirtingų disciplinų mokytojams nėra teisingos. Be to, nekontaktinėms
valandoms priklauso ir bendruomeninė mokyklos veikla. Iki etatinio
apmokėjimo darbuotojui galima buvo apmokėti iki 3,5 valandų per savaitę
nekontaktinių valandų. Dabar tokia veikla gali apimti iki trečdalio
etato. Tačiau juk pagrindinis mokytojo tikslas -- mokyti vaikus, ugdyti
jaunąją Lietuvos kartą. Šiuo atveju galima teigti, kad 30\% savo darbo
laiko mokytojas telkiasi ne į ugdymo procesą, o į šalutinius dalykus,
tokius, kaip mokyklos renginių organizavimas, kurie nepadeda mokiniams
tobulinti įgūdžių. Negana to, suskaičiuoti valandas, skirtas tokiai
veiklai, kaip sakė S. Jurkevičius, yra sudėtinga ir tai labai
smulkmeniškas darbas, atimantis daug laiko. Be to, tokį darbą sunku
prižiūrėti, suskaičiuoti jam tenkantį laiką, tad tenka pasitikėti
darbuotojo žodžiu. Bet tokiomis iniciatyvomis vis dėlto galėtų užsiimti
tie mokytojai, kurių laikas skiriamas pamokų pasiruošimui bei rezultatų
vertinimui nėra didelis (menų disciplinos, kūno kultūra). Etatinis darbo
modelis buvo priimtas 2018 metų Seimo vasaros sesijos pačioje pabaigoje
-- birželio 29 d. Nuo šios dienos buvo likę kiek daugiau nei du mėnesiai
iki įstatymo įsigaliojimo pradžios. Prieš mokslo metų pradžią likus
dviem savaitėms buvo skubama pasirašyti naujas darbo sutartis su
pedagogais, daugelyje mokyklų kilo lengvas chaosas, mokyklų vadovai
nebuvo perpratę atlyginimų perskaičiavimo bei etato sudarymo tvarkos.
Mokytojams buvo kilę daug klausimų bei baimių, kad bus nesudarytas
pilnas etatas, kad atlyginimai vis dėlto mažės. Tokiam svarbiam
pertvarkymui, paaiškinimui, kas keisis ir kaip viskas bus ministerija
tikrai galėjo skirti daugiau laiko. Kaip ir minėta, viena iš pagrindinių
baimių buvo, kad atlyginimai nedidės. Lentelės duomenys parodo, kad neto
atlyginimas mokytojams didėjo.

\begin{longtable}[]{@{}lr@{}}
\toprule
Laikotarpis & Neto atlyginimas\tabularnewline
\midrule
\endhead
2015K1 & 598.6\tabularnewline
2015K2 & 610.3\tabularnewline
2015K3 & 632.5\tabularnewline
2015K4 & 634.3\tabularnewline
2016K1 & 634.9\tabularnewline
2016K2 & 648.9\tabularnewline
2016K3 & 649.8\tabularnewline
2016K4 & 685.9\tabularnewline
2017K1 & 685.1\tabularnewline
2017K2 & 689.9\tabularnewline
2017K3 & 689.2\tabularnewline
2017K4 & 747.0\tabularnewline
2018K1 & 723.6\tabularnewline
2018K2 & 730.1\tabularnewline
2018K3 & 739.1\tabularnewline
2018K4 & 843.5\tabularnewline
2019K1 & 857.0\tabularnewline
Bendrą mokytoj & ų neto atlyginimų kaitą suprantamiau įvertinti padeda
grafikas:\tabularnewline
{[}{]}(Darbas\_fil &
es/figure-latex/unnamed-chunk-3-1.pdf)\tabularnewline
Etatinis apmok & ėjimas įsigaliojo nuo rugsėjo 1 d., tai atlyginimų
pokytį parodo 2018 metų ketvirto ketvirčio duomenys. Atlyginimai
atskaičius mokesčius palyginus su metų pradžia kilo 16,6\%. 2019 metų
pirmą ketvirtį neto atlyginimas dar šiek tiek paaugo. Tada kyla
klausimas, ar šis kilimas nebuvo nulemtas mokytojų skaičiaus
mažinimu?\tabularnewline
{[}{]}(Darbas\_fil &
es/figure-latex/unnamed-chunk-4-1.pdf)\tabularnewline
\bottomrule
\end{longtable}

Nuo mokslo metų pabaigos iki naujų metų pradžios, kai įsigaliojo
etatinis mokytojų apmokėjimas, iš darbo išėjo 400 mokytojų, kurie sudarė
1,5\% nuo birželio mėnesį dirbančiųjų. Lygiai tokią pačią tendenciją
buvo galima pastebėti ir 2017 metais, kai iš darbo išėjo 500 mokytojų,
kurie sudarė 1,8\% nuo dirbančių iki mokslo metų pabaigos. Vadinasi,
mokytojų etatinis apmokėjimas vis dėlto pagerino mokytojų atlyginimus.
Tačiau ne visiems. (Nagrockienė (2018)) Akivaizdu, kad labiausia
atlyginimai kilo mažesnių mokyklų mokytojams, kurie anksčiau
nesurinkdavo daug kontaktinių valandų, o po etatinio apmokėjimo etatą
susirenka pasiimdami valandų už mokyklos bendruomeninę veiklą.
Mažiausiai atlyginimai kilo didesnių mokyklų, o ypač gimnazijų
mokytojams, kuriems nebeliko 5--20\% priedo. Kylantis mokytojų
nepasitenkinimas privedė iki mėnesį laiko trukusio mokytojų streiko.
(Antanavičius (2018)). Streiko metu buvo atstatydinta Švietimo ir mokslo
ministrė Jurgita Petrauskienė. Po šio įvykio pavyko ministerijai ir
profsąjungoms rasti kompromisinį variantą: buvo paskirti 185 milijonai
eurų švietimui, nors valdžia buvo paskaičiavusi, kad profsąjungos
reikalavimams įvykdyti reikės 300 milijonų eurų. Taigi kur nueina
pinigai, skirti švietimui? Pagal 2016 metų duomenis Lietuva viršijo ES
vidurkį švietimo reikmėms skaičiuojant pagal proc. nuo BVP. Lietuva
skyrė 5,2\% nuo BVP, kai ES vidurkis -- 4,7\% (Komisija (2018)). Žinoma,
Skandinavijos šalys skiria dar daugiau lėšų švietimui (Komisija (2018)).
Švedija, Suomija, Danija skiria nuo 6\% iki 7\% BVP. Tačiau pastebima,
kad Lietuvoje lėšos panaudojamos neefektyviai. Viena iš neefektyvumo
priežasčių -- pastatų išlaikymas. Ir tai nieko nuostabaus, kai mokyklos
yra nepilnos. Žurnale „Reitingai`` pateikiamas Gintaro Sarafino
straipsnis „Turtingoji Lietuva: „auksiniai`` vaikai ir tūkstančiai už
dvejetus``(Sarafinas (2019)) straipsnio autoriaus pateikia paradoksalią
(net keliomis prasmėmis) situaciją. Pirmiausia jis atkreipia dėmesį į
tai, kad to pačio rajono mokyklos finansavimas vienam mokiniui per metus
gali skirtis net kelis kartus. Bene didžiausias matomas skirtumas tarp
Mažeikių rajono Šerkšnėnų mokyklos daugiafunkcinio centro, kuris vienam
mokiniui skiria 1893 eurus per metus bei Mažeikių rajono Krakių
pagrindinės mokyklos, skiriančios mokiniui po 7140 eurų per metus. Taigi
šių mokyklų lėšos vienam mokiniui skiriasi 3,77 karto. Daugiausia lėšų
mokiniui tenka Vilniaus rajono Sužionių pagrindinei mokyklai -- 7263
eurai. Akivaizdu, kad tokios didelės sumos susidaro dėl to, kad mokinių
mokyklose nedaug, tačiau daug finansų pareikalauja elektra, šildymas,
skirtingų disciplinų mokytojų atlyginimai. Antras paradoksas -- tose
mokyklose, kur mokiniams skiriama per metus daugiausia lėšų, jie
pasiekia vienų iš prasčiausių rezultatų. Vadinasi, į mažesnes rajono
mokyklas sunku pritraukti kvalifikuotų bei motyvuotų specialistų. Yra
savivaldybių, kurios pasiryžo tokią problemą spręsti. Kaip pavyzdys
įvardijama Druskininkų savivaldybė, kuri optimizavo mokyklų tinklą.
Druskininkų meras Ričardas Malinauskas (oh well) džiaugiasi to
rezultatu, nes neliko pustuščių didelių mokyklų pastatų, kuriuos reikėtų
šildyti, taip sutaupytos lėšos nukreipiamos kitur. Tiesa, jis
pripažįsta, kad toks sprendimas iš pradžių sulaukė nepopuliarumo. Tiek
tėvams, tiek mokiniams atrodo nepatraukli mintis, kad šalia buvusi
mokykla užsidaro, o vietoj jos laukia važiavimas į toliau esančią
mokyklą geltonuoju autobusiuku. Tačiau, anot mero, po kurio laiko tėvai
džiaugiasi, nes jų vaikai gauna geresnes sąlygas mokytis. Žinoma, tai
nereiškia, kad visas mažesnes mokyklas reikia staiga imti ir uždaryti.
Tiek vaikams, tiek jų tėvams būtų patogiausia, jeigu pradinės mokyklos
būtų šalia jų gyvenamos vietos (bet juk tai galėtų būti ir kitų mokyklų
skyriai, taip būtų sutaupoma išlaidų, kurios būtų skirtos
administracijai). Išskirtinai mažas pradines mokyklas būtų galima
uždaryti tik tuo atveju, jeigu savivaldybė užtikrintų patogų moksleivių
nuvežimą/parvežimą iš mokyklos su geltonaisiais autobusiukais, ypač ten,
kur retai važinėja viešasis transportas, taip nesukeliant nei tėvams,
nei vaikams papildomų rūpesčių, o ir didesnė tikimybė būtų gauti geresnį
išsilavinimą. Nors Lietuva skiria didesnį nei ES vidurkis finansavimą
pagal procentus nuo BVP, vis dėlto Lietuvos rezultatai tarptautiniu
mastu neatrodo labai gerai. 2015 metais atliktas tarptautinis
penkiolikmečių tyras PISA parodo, kad Lietuva tarp visų EBPO šalių
rikiuojasi apie vidurį. Na štai iš gamtamokslinio raštingumo Lietuva
užima 36 vietą, šalia Islandijos, Kroatijos bei Italijos. Už jos iš ES
šalių yra tik Malta, Graikija, Slovakija, Bulgarija bei Kipras. O mūsų
kaimynai estai yra treti. Pagal skaitymo gebėjimus Lietuva yra 38,
pasiekusi panašių rezultatų kaip Izraelis, Graikija ir Vengrija, už
nugaros palikusi tik 7 kitas ES šalis. Pagal matematinį raštingumą
Lietuva taip pat 36, palikdama už savęs taip pat 7 ES šalis. Na, o
Suomija laikosi pirmame penketuke. Kur jos paslaptis? Iš tikrųjų ši
šalis turi labai daug panašumų su Lietuva. Pamokos lygiai tiek pat
trunka 45 minutes. Kaip bebūtų keista, ten daugiausia įtakos mokykloms
turi vietinė valdžia. Ji taip pat sprendžia, kurią rugpjūčio savaitę
prasidės mokslo metai (tai priklauso nuo tos vietovės klimato).
Pasidomėjus darosi akivaizdu, LR Švietimo ir mokslo ministerija visais
būdais stengiasi neatsilikti nuo Suomijos. Ministro įsakymu net buvo
prailginti mokslo metai Lietuvoje. Ir iš tikrųjų, Suomijoje mokslo metai
baigiasi paskutinėmis gegužės ar pirmomis birželio dienomis. Iš viso
Suomijos vaikai turi 10--11 vasaros atostogų savaičių ir dar maždaug 6
savaites atostogų mokslo metų eigos metu. Lietuvoje šie mokslo metai
baigiasi prieš pat Jonines, taigi vasaros atostogų trukmė taip pat bus
apie 10 savaičių. Lietuviai dar turi 5 savaites atostogų mokslo metų
metu ir dar gali pakliūti 3 valstybinių švenčių dienos. Ir iš tikrųjų
toks dirbtinis panašumas jau daugiau negu bado akis. Lietuvoje per
papildomai pridėtas mokslo metų savaites mokiniai arba praktiškai nieko
neveikia (tarkim žiūri jų išsirinktus filmus), arba jau nebeina į
mokyklą, ką galima būtų pavadinti savivale, o vyresni mokiniai pradeda
vasaros darbus ir tik mažumoje mokyklų vyksta mokymosi procesas. Tiesa,
ministerija nepamiršo pasirūpinti mokinių mokslo metų ilgumu, bet
pamiršo vieną paprastą dalyką -- dauguma Lietuvos mokyklų yra senos
statybos ir neturi kondicionierių. Mažiausiai antrus metus nuo pavasario
pabaigos iki pat mokslo metų pabaigos Lietuvą „užgriūna`` dideli
karščiai. Pietiniuose kabinetuose karštis siekia 38 laipsnius karščio.
Ministerija į tai atsako, kad organizuokit veiklą lauke (na žinoma, nes
ten vėsiau, kokie 35 laipsniai karščio). Galų gale mokyklos trumpino
pamokų laiką, o tai realiai reiškia, kad į ją eiti net neapsimoka (ypač
tiems, kurie gyvena toliau nuo mokyklos, nes jų kelionės į mokyklą
laikas tampa beveik lygus bendram pamokų laikui tą dieną). O be to, tai
gali netgi pakenkti sveikatai, ypač jeigu važiuojama neatnaujintu
viešuoju transportu, o nuo stotelės iki namų tenka paėjėti nei daug nei
mažai -- kelis kilometrus -- vidudienį per patį didžiausią karštį
(asmeninė patirtis). EBPO pastebi, kad suomiai vaikus išleidžia bene
vėliausiai -- 7 metų. Ir jų mokymosi trukmė per metus yra viena
trumpiausių (tiek mokslo metų trukmė, tiek pamokų kiekis per savaitę).
Suomija dar pasižymi įdomiu dalyku -- jie daugiausia iš visų vidurinio
ugdymo pakopų laiko skiria pradiniam ugdymui (OECD (2018)). Jam per
savaitę skiriama vidutiniškai 18--19 valandų. Pagrindiniam ugdymui
skiriama (5--8 klasės) 16--17 valandų, o vyresniems mokiniams tenka apie
16 valandų. Lietuvoje analogiškai skiriama 23, 28 bei nuo 28 iki 35
pamokų per savaitę (nuoroda
{[}\url{https://www.smm.lt/uploads/documents/svietimas_pagrindinis_ugdymas_spec/BUP\%20\%C4\%AEsakymas\%202017-2019\%20m.pdf}{]}).
Lietuvoje pastaruosius kelerius metus ne tik dvyliktokai laiko
egzaminus, bet ir ketvirtokai, aštuntokai rašo standartizuotus testus.
Suomijoje tokio dalyko nėra. Ir nesunku paaiškinti kodėl. Tokiu atveju
mokytojas stengiasi išmokyti vaikus, kad jie išlaikytų testą, taigi jis
ruošiamas būtent testo išlaikymui, o tai neturi prasmės, nes mokinys
turi būti skatinamas individuliai mąstyti, o ne atlikti užduotis taip,
kad įtiktų vertintojui. Pripažinkim, Lietuvoje su tuo yra didelė
problema. Didžiausias iššūkis kiekvienam dvyliktokui -- kaip išlaikyti
lietuvių kalbos ir literatūros VBE. Realiai ketverius metus (nuo 9
klasės) mokiniai ruošiami rašyti ir mąstyti taip, kad įtiktų
vertintojams, nors iš tikrųjų neįmanoma tiksliai pasakyti, o ką galvoje
turėjo XX amžiaus rašytojas Antanas Škėma. Tačiau mokyklose mus to moko.
Dar vienas svarbus dalykas, kad Suomijoje mokytojai turi turėti magistro
laipsnį. Tai būtina, nes mokytojai turi ganėtinai daug laisvės ruošdami
ugdymo programą (education (2018)). Mokytojo profesija Suomijoje yra
ypač gerbiama, į ją sunku įstoti, tai byloja apie mokytojo profesijos
prestižą, kas leidžia pritraukti tik labiausiai motyvuotus būsimus
mokytojus. Suomijoje skiriama daug laiko kvalifikacijos kėlimo kursams.
Lygiai to pačio reikėtų ir Lietuvoje, nes kai kurie mokytojai tikrai yra
atsilikę su pamokų dėstymo naujovėmis, būdais, kuriuos galima pasitelkti
sudominant mokinius pamokos medžiaga. Viena ne mažiau svarbi ypatybė
Suomijos mokyklose, kad mokytojai tuos pačiu vaikus moko ne mažiau kaip
šešerius metus. Ir atrodo, ar yra didelis skirtumas kiek laiko tas pats
mokytojas moko vaikus? Ir vis dėlto yra. Tame pačiame žurnale
„Reitingai`` yra 2019 metų gimnazijų reitingas (Jonė Kučinskaitė (2019))
šalia pateikiamos Lietuvos švietimo profesinės sąjungos vadovo Audriaus
Jurgelevičiaus mintys, kad mokyklų tinklo pertvarka davė neigiamų
rezultatų. Panaikinus daugumą vidurinių mokyklų mokiniams per visą
mokymosi etapą tenka praeiti mažiausiai dvi arba netgi tris mokyklas
(pradinę, progimnaziją (kartais šios dvi būna apjungtos) ir gimnaziją).
Anot A. Jurgelevičiaus pirmasis pusmetis būna skirtas prisitaikyti prie
mokytojų, naujos aplinkos bei klasės draugų, reikalavimų, mokyklos
kultūros bei vertybių ir šis laikas moksleiviams būna sunkus, tad kenčia
mokymosi rezultatai bei pats mokymasis nebėra toks produktyvus. Kaip
viena iš tokios pertvarkos blogybių dar minima situacija, kai mokytojai
nuleidžia rankas matydami nemotyvuotus mokinius, neranda priėjimo prie
jų bei nesiekia įkvėpti mokytis ir tarsi nusimeta atsakomybę nuo savęs,
nes juk mokinys vis tiek išeis iš šios mokyklos į kitą ir tegu kiti
specialistai dirba. Kai mokinys išeina į kitą mokyklą dar ir nereikia
jausti atsakomybės prieš savo kolegas, dirbančius toje pačioje
mokykloje. Dar Lietuvos mokytojai yra per daug susitelkę į savo dalyką,
kaip reikia „išeiti`` programą, o ne į asmeninius vaikų pasiekimus. Kai
mokinys kažko neišmoksta ar prastai parašo kontrolinį tik vienetai
mokytojų leidžia perrašyti darbą, visiems kitiems nelabai svarbu, kad su
mokiniu reikia individualiai padirbėti tam, kad jis suprastų išdėstytą
temą. Juk visada dalykai būna tarpusavyje susiję: neišmokę vieno
mokiniai negebės suprasti kito, o taip ypač yra su disciplinomis, kurios
mokiniams sekasi sunkiau -- matematika, fizika, chemija. Kai kurie
teigia, kad toks mokymo metodas stabdytų kitų vaikų ugdymo procesą. Bet
laikas, skirtas mokiniams pakartoti, geriau suvokti temą, galėtų būti
skirtas po pamokų. Tokiu atveju būtų išspręsta problema dėl
korepetitorių: dabar jiems finansų negali skirti šeimos, neturinčios tam
pinigų. Sprendžiant visas problemas yra svarbiausia tartis su švietimo
bendruomene ir kartu ieškoti palankiausių sprendimų, kurie padėtų
mokiniams pasiekti kuo geresnių rezultatų. Siekiant išspręsti problemas
gali ne viskas keistis į gerą, bus padaryta daug klaidų, svarbiausia
nebijoti pripažinti jas padarius bei taisyti esamą situaciją, kai kada
reikės imtis ir nepopuliarių sprendimų, svarbiausia, kad jie būtų
tinkamai iškomunikuoti bendruomenei ir kad jai būtų pakankamai laiko
naujoves suvokti bei jas priimti. Na ir pabaigai, giriama Suomijos
švietimo sistema neatsirado per vieną dieną, jai reikėjo daug daugiau
laiko, be to ši šalis neturėjo 50 metų trukusios sovietinės okupacijos.
Tad galima tikėtis, kad po ilgesnio laiko tarpo Lietuvą taip pat kaip
pavyzdį naudos kitos Europos bei pasaulio valstybės.

\subsection*{Literatūra}\label{literatura}
\addcontentsline{toc}{subsection}{Literatūra}

\hypertarget{refs}{}
\hypertarget{ref-ugniusantanaviux10dius2018}{}
Antanavičius, Ugnius. 2018. ``15 Min Paaiškina: Kodėl Streikuoja
Mokytojai Ir Ko Jie Reikalauja?'' \emph{„15min``}.

\hypertarget{ref-finnishnationalagencyforeducation2018}{}
education, Finnish national agency for. 2018. ``Teacher Education.''

\hypertarget{ref-jonux117kuux10dinskaitux117rasakairienux1172019}{}
Jonė Kučinskaitė, Rasa Kairienė. 2019. ``„2019 M. Gimnazijų
Reitingas``.'' \emph{„Reitingai``}.

\hypertarget{ref-sauliusjurkeviux10dius2018}{}
Jurkevičius, Saulius. 2018. ``Saulius Jurkevičius. Kodėl Būtina
Sustabdyti Mokytojų Etatinio Apmokėjimo įvedimą?'' \emph{„Delfi``}.

\hypertarget{ref-europoskomisija2018}{}
Komisija, Europos. 2018. ``Šalies Ataskaita. Lietuva 2019.''

\hypertarget{ref-ingridanagrockienux1172018}{}
Nagrockienė, Ingrida. 2018. ``Pažadai Nesutilpo į Mokytojo Piniginę.''
\emph{„Sekundė``}.

\hypertarget{ref-oecd2018}{}
OECD. 2018. ``Education at a Glance. Finland.''

\hypertarget{ref-profesux105jungos2017}{}
Profesąjungos. 2017. ``Etatinio Darbo Apmokėjimas.''

\hypertarget{ref-gintarassarafinas2019}{}
Sarafinas, Gintaras. 2019. ``„Turtingoji Lietuva: „Auksiniai`` Vaikai Ir
Tūkstančiai Už Dvejetus``.'' \emph{„Reitingai``}.

\hypertarget{ref-lrseimas2016}{}
Seimas, LR. 2016. ``Nutarimas Dėl Lietuvos Respublikos Vyriausybės
Programos.''

\hypertarget{ref-ux161mm2011}{}
ŠMM. 2011. ``Švietimo įstaigų Darbuotojų Ir Kitų įstaigų Pedagoginių
Darbuotojų Darbo Apmokėjimo Tvarkos Aprašas.''

\hypertarget{ref-ux161mm2018}{}
---------. 2018. ``Patobulintas Etatinio Apmokėjimo Modelis.''

\hypertarget{ref-auux161raux17eidux17eiux16bnienux1172018}{}
Židžiūnienė, Aušra. 2018. ``Etatinio Darbo Apmokėjimo Sėkmės Ir
Nesėkmės\ldots{}.'' \emph{„Švietimo Naujienos``}.


\end{document}
